\documentclass[a4paper,12pt]{article}
\usepackage[utf8]{inputenc} %Codificacion utf-8
\usepackage{fancyhdr, float, graphicx, caption}
\usepackage{graphicx}
\usepackage{multicol}

\pagestyle{fancy}
\renewcommand{\figurename}{Figura}
\renewcommand\abstractname{\textit{Abstract}}

\fancyhf{}
\fancyhead[LE,RO]{\textit{Mini Virtual Machine}}
\fancyfoot[RE,CO]{\thepage}

\title{
	%Logo UNR
	\begin{figure}[!h]
		\centering
		\includegraphics[scale=1]{unr.png}
		\label{}
	\end{figure}
	% Pie Logo
	\normalsize
		\textsc{Universidad Nacional de Rosario}\\	
		\textsc{Facultad de Ciencias Exactas, Ingenier\'ia y Agrimensura}\\
		\textit{Licenciatura en Ciencias de la Computaci\'on}\\
		\textit{Arquitectura del Computador}\\
	% Título
	\vspace{30pt}
	\hrule{}
	\vspace{15pt}
	\huge
		\textbf{Mini Virtual Machine}\\
	\vspace{15pt}
	\hrule{}
	\vspace{30pt}
	% Alumnos/docentes
	\begin{multicols}{2}
	\raggedright
		\large
			\textbf{Alumnos:}\\
		\normalsize
			BORRERO, Paula (P-4415/6) \\
			CRESPO, Lisandro (C-6165/4) \\
			MISTA, Agust\'in (M-6105/1) \\
	\raggedleft
		\large
			\textbf{Docentes:}\\
		\normalsize
			RUIZ, Esteban \\
			FEROLDI, Diego \\
			SCANDOLO, Leonardo\\
			BERGERO, Federico\\
	\end{multicols}
}

\begin{document}
\date{9 de Abril de 2015}
\maketitle


\pagebreak

\section{Introducci\'on}

	En \'este trabajo se pretende explorar los detalles de ejecuci\'on e implementaci\'on de una m\'aquina abstracta b\'asica, como as\'i tambi\'en de un lenguaje ensamblador que la misma pueda ejecutar de manera nativa.
	
	Para ello desarrollamos una con las siguientes caracter\'isticas:
\begin{itemize}
\item Posee 8 registros de 4 bytes. Ellos son: ZERO, siempre vale cero. PC indica qu\'e instrucci\'on es la pr\'oxima a ejecutar (de la lista de instrucciones). SP apunta al tope de la pila. R0,..,R3 son registros de uso general y FLAGS contiene bits de estado (resultado de operaciones, comparaciones, etc).
\item Posee una \'unica memoria en la cual se puede escribir y leer, mediante instrucciones MOV o POP/PUSH.
\item Su juego de instrucciones es el siguiente:
\begin{itemize}
\item NOP: no realiza ninguna operaci\'on.
\item MOV: mueve un valor de/hacia registro/memoria/constante.
\item SW: guarda el valor de src en la direcci\'on de memoria dst.
\item LW: carga en dst el valor de la memoria apuntada por src.
\item PUSH: escribe el argumento src en el tope de la pila, decrementando SP por 4.
\item POP: escribe en el argumento dst el tope de la pila, incrementando el SP por 4.
\item PRINT: imprime por pantalla el entero descripto en el argumento src.
\item READ: lee un entero y lo guarda en el argumento de destino.
\item ADD/SUB: suma/resta los argumentos src y dst y lo guarda en dst.
\item AND/OR/XOR: realiza un bitwise and/or/xor entre los dos operandos y lo guarda en dst.
\item LSH/RSH: realiza un corrimiento de src bits a izquierda/derecha de dst y lo guarda en dst.
\item MUL/DIV: multiplica/divide los argumentos src y dst y lo guarda en dst.
\item CMP: compara src y dst y setea los bits correspondientes en el registro FLAGS.
\item JMP: salta a la instrucci\'on n\'umero dst de la lista de instrucciones.
\item JMPE: salta a la instrucci\'on n\'umero dst si la bandera de igualdad est\'a seteada.
\item JMPL: salta a la instrucci\'on n\'umero dst si la bandera de menor est\'a seteada.
\item CALL: guarda el registro contador de programa en el stack y salta al label apuntado por src.
\item RET: borra el registro contador de programa del stack y salta a la direcci\'on apuntada por el mismo.
\item HLT: termina el programa.
\end{itemize}
\end{itemize}

Todos sus operandos son de 4 bytes.
	
\section{Problemas encontrados}

\begin{enumerate}
\item 
En un principio nos enfocamos en entender el c\'odigo que tendr\'iamos que modificar. Primero analizamos los miembros de la estructura Operand para entender mejor su funci\'on. As\'i llegamos a la conclusi\'on de que el entero \emph{val} representaba:
 
	\textbf{a)} El valor propiamente dicho del operando si \'este era de tipo inmediato (el miembro type era \emph{IMM}),
	
	\textbf{b)} el \'indice dentro del arreglo de registros en donde se encontraba el operando si era un registro (\emph{REG}),
	
	\textbf{c)} el \'indice / 4 dentro del arreglo que representa a la memoria de la m\'aquina si era un valor de memoria (\emph{MEM}),
	
	\textbf{d)} una direcci\'on en caso de que sea una etiqueta (\emph{LABELOP})\\

Adem\'as, en el caso de que el operando sea un label, en el miembro lab de la estructura se guarda una cadena con su nombre. 
\item
Luego comenzamos a definir cada una de las funciones tratando de tener cuidado sobre qu\'e tipos de operadores \'ibamos a permitir que se pudieran ejecutar. Para ello usamos como ejemplo las funciones de assembler y tratamos de llegar a un consenso entre los integrantes del grupo.
\item
Tambi\'en tuvimos que decidir c\'omo utilizar las flags. Nos centramos sobre todo en definir bien en qu\'e funciones iba a hacerse uso de la flag equals y en cu\'ales de zero, ya que podr\'iamos haber usado solo zero y, en el caso de las comparaciones, hacer una resta entre los operandos y si la misma ten\'ia como resultado 0 entonces prend\'iamos la flag zero para indicar la igualdad. Concordamos en que la comparaci\'on (\emph{CMP}) pod\'ia encender solo las flags lower y equals y las operaciones aritm\'eticas y de bits la flag zero.
\item 
A medida que \'ibamos haciendo funciones hac\'iamos varios tests para comprobar que se la m\'aquina se comportaba como pretend\'iamos pero, antes de empezar el test2 dado, nos pareci\'o conveniente implementar instrucciones de operadores de bits como caracter\'istica adicional. Para esto necesit\'abamos modificar los archivos de Bison y Flex (analizador sint\'actico y analizador l\'exico respectivamente) para añadir los nuevos comandos. Lo cambios en estos archivos los realizamos buscando informaci\'on en Internet y usando como base l\'ineas escritas para otros comandos ya existentes.
\item
Una vez que implementamos la operaci\'on \emph{READ}, notamos que no funcionaba como era esperado, ya que para que la m\'aquina virtual leyera el archivo que deb\'ia ejecutar, \'este era pasado redirigi\'endolo a la entrada est\'andar de la misma, lo cual evitaba que se use el teclado como entrada est\'andar para leer un valor.
Est\'a claro que una m\'aquina virtual que no pueda leer valores por teclado no resulta muy \'util. Afortunadamente, la implementaci\'on del parser tiene flexibilidad en \'este sentido. Su entrada es obtenida desde lo apuntado por su variable externa \emph{yyin}, por lo que ahora nuestro archivo a ejecutar por la m\'aquina es pasado como argumento, y solo hace falta abrirlo y asignarlo a \emph{yyin}.
\item
Quisimos agregar tambi\'en los comandos \emph{CALL} y \emph{RET} a nuestra m\'aquina virtual. Lo hicimos tomando como referencia las funciones de assembler de la arquitectura x86\_64. Al principio no andaba el test que hab\'iamos hecho para probar estas funciones pero despu\'es vimos que era porque nos hab\'iamos olvidado de incluir a la funci\'on \emph{CALL} en la funci\'on \emph{processLabels}. 
\end{enumerate}

\section{Descripci\'on de los casos de prueba}

Para comprobar el correcto funcionamiento de nuestra m\'aquina escribimos varias pruebas complementarias a las requeridas. A continuaci\'on se da una breve descripci\'on de los casos de prueba requeridos junto con la de los que consideramos oportuno añadir:
\begin{itemize}
\item
\textbf{Nombre:} test1.asm\\
\textbf{Descripci\'on:} lee un entero e imprime el valor absoluto de dicho n\'umero.
\item
\textbf{Nombre:} test2.asm\\
\textbf{Descripci\'on:} imprime en pantalla la cantidad de bits de un entero dado.
\item
\textbf{Nombre:} test3.asm\\
\textbf{Descripci\'on:} dado un arreglo de elementos guardado en la pila, imprime en pantalla la suma de sus elementos.
\item
\textbf{Nombre:} test\_read.asm\\
\textbf{Descripci\'on:} lee un entero, lo imprime en pantalla y repite la operaci\'on a menos que se le pase un 0, en cuyo caso termina su ejecuci\'on.
\item
\textbf{Nombre:} test\_stack.asm\\
\textbf{Descripci\'on:} hace push de 5 enteros a la pila y luego los extrae mediante pop. Se us\'o para probar las funciones de pila.
\item
\textbf{Nombre:} test\_zero\_flag.asm\\
\textbf{Descripci\'on:} ejecuta algunas operaciones aritm\'eticas y de bits, a fin de probar el funcionamiento de la flag \emph{ZERO}
\item
\textbf{Nombre:} test\_call\_ret.asm\\
\textbf{Descripci\'on:} llama a una funci\'on \emph f mediante \emph{CALL/RET}. Se us\'o para testear el funcionamiento de \'estos operadores.
\end{itemize}

\section{Posibles extensiones}

\begin{enumerate}
\item Añadir flags adicionales como \emph{parity flag}, \emph{carry flag}, \emph{sign flag} y \emph{direction flag}.
\item Agregar instrucciones de manejo de cadenas.
\item Añadir registros, instrucciones y operaciones para n\'umeros de punto flotante.
\item Definir una convenci\'on de llamadas.
\end{enumerate}

\section{Conclusi\'on}

	Podemos concluir que, si bien la m\'aquina virtual que implementamos tiene notables limitaciones, \'esta posee una buena escalabilidad en cuanto a sus capacidades, pudi\'endose extender de manera relativamente sencilla, dot\'andola de capacidades m\'as interesantes.

\vspace{\fill}

\begin{multicols}{3}

	\hrule
	\vspace{5pt}
	BORRERO, Paula\\
	\linebreak

	\hrule
	\vspace{5pt}
	CRESPO, Lisandro \\
	\linebreak

	\hrule
	\vspace{5pt}
	MISTA, Agust\'in \\
	
\end{multicols}
\end{document}